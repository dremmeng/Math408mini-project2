% LaTeX Article Template - customizing page format
%
% LaTeX document uses 10-point fonts by default.  To use
% 11-point or 12-point fonts, use \documentclass[11pt]{article}
% or \documentclass[12pt]{article}.
\documentclass{article}

% Set left margin - The default is 1 inch, so the following 
% command sets a 1.25-inch left margin.
\setlength{\oddsidemargin}{0.25in}

% Set width of the text - What is left will be the right margin.
% In this case, right margin is 8.5in - 1.25in - 6in = 1.25in.
\setlength{\textwidth}{6in}

% Set top margin - The default is 1 inch, so the following 
% command sets a 0.75-inch top margin.
\setlength{\topmargin}{-0.25in}

% Set height of the text - What is left will be the bottom margin.
% In this case, bottom margin is 11in - 0.75in - 9.5in = 0.75in
\setlength{\textheight}{8in}
\usepackage{fancyhdr}
\usepackage{float}
\usepackage{mathtools}
\usepackage{amsmath}
\usepackage{amssymb}
\setlength{\parskip}{5pt} 
\pagestyle{fancyplain}
% Set the beginning of a LaTeX document
\begin{document}

\lhead{Drew Remmenga MATH 408}
\rhead{Project \#2}
%\lhead{Independent Study}
%\rhead{R Lab}

\begin{enumerate}

\item 
The function:
\begin{equation*}
\begin{split}
f(t,u) = u^{2}e^{-t^{2}}sin(t)
\end{split}
\end{equation*}
is Lipschitz continous in $u$ on:
\begin{equation*}
\begin{split}
 \mathcal{D} =\{(t,u): t\in \mathbb{R}, u \in [0,2]\}
\end{split}
\end{equation*}
if 
\begin{equation*}
\begin{split}
\exists L \in \mathbb{R_{+}} : \left|f(t,u) -f(t,u^{*}) \right| & \leq L \left| u - u^{*} \right| \\
\left|  u^{2}e^{-t^{2}}sin(t) - u^{*2}e^{-t^{2}}sin(t)  \right|  &\leq L \left| u - u^{*} \right| \\
\left|  u^{2}- u^{*2}  \right| \left|e^{-t^{2}}sin(t)\right|  &\leq L \left| u - u^{*} \right| \\
\left|  u- u^{*}  \right| \left|u+u^{*}\right| \left|e^{-t^{2}}sin(t)\right|  &\leq L \left| u - u^{*} \right| \\
\end{split}
\end{equation*}
Since $f(t,u)$ is differentiable with respect to $u$ and bounded in $\mathcal{D}$ then we can set the Lipschitz constant:
\begin{equation*}
\begin{split}
L &= \max_{(t,u) \in \mathcal{D}} \left| f_{u}(t,u) \right| \\
& =  \max_{(t,u) \in \mathcal{D}} \left| 2ue^{-t^{2}}sin(t) \right| \\
&\approx 4*.397 \\
& \approx 1.588 \\
\end{split}
\end{equation*}
\item
\begin{enumerate}
\item
Since $f(u)$ is differentible at all points $u \in \mathbb{R}$, $f(u)$ is Lipshitz at every point $u \in \mathbb{R}$.
With L $\leq$ 2.
\item
Take Dirchlet's 'jagged' discontinuous function which is continuous nowhere and is differentiable nowhere but is nonetheless bounded by 1.
\end{enumerate}
\item
\begin{equation*}
\begin{split}
U  &=
\begin{bmatrix}
u_{1}& 0& 0& 0 \\
0 & u_{2} & 0 & 0 \\
0 & 0 & u_{3} & 0 \\
0 & 0 & 0 & u_{4}
\end{bmatrix} \\
U' &= \begin{bmatrix}
u_{2} & 0 & 0 & 0 \\
0 & -\frac{u_{1}}{(u_{1}^{2}+u_{3}^{2})^{3/2}} & 0 & 0 \\
0 & 0 &  u_{4} & 0 \\
0 & 0 & 0 & -\frac{u_{3}}{(u_{1}^{2}+u_{3}^{2})^{3/2}}
\end{bmatrix}
\end{split}
\end{equation*}
\end{enumerate}



\end{document}
